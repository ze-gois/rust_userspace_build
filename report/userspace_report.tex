\documentclass{beamer}
\usepackage[utf8]{inputenc}
\usepackage[T1]{fontenc}
\usepackage{lmodern}
\usepackage{listings}
\usepackage{xcolor}
\usepackage{graphicx}
\usepackage{hyperref}
\usepackage{minted}
\usepackage{amsmath}
\usepackage{booktabs}

% Beamer theme settings
\usetheme{Madrid}
\usecolortheme{default}
\setbeamertemplate{navigation symbols}{}
\setbeamertemplate{caption}[numbered]
\setbeamercolor{structure}{fg=blue!70!black}
\setbeamertemplate{itemize item}[circle]
\setbeamertemplate{itemize subitem}[triangle]

% Title page info
\title{Userspace: A Standard Library for Userspace Applications}
\subtitle{Rust Implementation for Operating System Interfaces}
\author{José Gois \\ \small{ze.gois.00@gmail.com}}
\institute{Federal University of Rio Grande do Norte (UFRN)}
\date{\today}

% Code style settings
\definecolor{codebg}{rgb}{0.95,0.95,0.95}
\definecolor{codeframe}{rgb}{0.8,0.8,0.8}
\definecolor{codenumber}{rgb}{0.5,0.5,0.5}
\definecolor{codestring}{rgb}{0.58,0,0.82}
\definecolor{codekeyword}{rgb}{0,0,0.8}
\definecolor{codecomment}{rgb}{0,0.6,0}

\lstset{
  basicstyle=\footnotesize\ttfamily,
  backgroundcolor=\color{codebg},
  frame=single,
  framesep=2pt,
  framerule=0.4pt,
  rulecolor=\color{codeframe},
  numbers=left,
  numberstyle=\tiny\color{codenumber},
  numbersep=5pt,
  breaklines=true,
  showspaces=false,
  showstringspaces=false,
  stringstyle=\color{codestring},
  keywordstyle=\color{codekeyword}\bfseries,
  commentstyle=\color{codecomment}\itshape,
  captionpos=b,
  tabsize=2,
}

\lstdefinelanguage{Rust}{
  sensitive=true,
  keywords={
    self, fn, pub, struct, enum, const, static, mut, let, if, else, for, while,
    match, mod, use, extern, trait, impl, where, unsafe, type, move, as, in, ref, return, true, false,
    u8, u16, u32, u64, usize, i8, i16, i32, i64, isize, f32, f64, bool, str, char
  },
  morecomment=[l]{//},
  morecomment=[s]{/*}{*/},
  morestring=[b]",
  morestring=[b]',
}

\begin{document}

\begin{frame}
  \titlepage
\end{frame}

\begin{frame}{Outline}
  \tableofcontents
\end{frame}

\section{Introduction}

\begin{frame}{Introduction}
  \begin{itemize}
    \item \textbf{Userspace} is a Rust-based standard library for userspace applications
    \item Provides a cross-platform abstraction layer for operating system interfaces
    \item Built with a \texttt{no\_std} environment in mind
    \item Focused on memory safety, portability, and performance
    \item Aims to provide a consistent API across different architectures and operating systems
  \end{itemize}
\end{frame}

\begin{frame}{Project Objectives}
  \begin{itemize}
    \item Create a modern alternative to libc for systems programming
    \item Ensure memory safety through Rust's ownership model
    \item Abstract architecture-specific details (currently targeting x86\_64)
    \item Implement ELF format parsing and handling
    \item Provide low-level memory management utilities
    \item Enable cross-platform development of userspace applications
  \end{itemize}
\end{frame}

\section{Methods}

\begin{frame}{Project Architecture}
  The library is structured in a modular way:
  \begin{itemize}
    \item \textbf{Core Library}: Basic types, traits, and utilities
    \item \textbf{Target Layer}: Architecture and OS-specific implementations
    \item \textbf{Memory Management}: Stack handling, page allocation
    \item \textbf{File Formats}: ELF parsing and interpretation
    \item \textbf{Error Handling}: Custom result types
  \end{itemize}
\end{frame}

\begin{frame}{Technology Stack}
  \begin{itemize}
    \item \textbf{Rust}: Memory-safe systems programming language
    \item \textbf{No Standard Library}: Using \texttt{\#![no\_std]} for bare-metal compatibility
    \item \textbf{Rust Features}:
      \begin{itemize}
        \item Generic const expressions
        \item Trait implementations
        \item Unsafe code blocks for low-level operations
      \end{itemize}
    \item \textbf{Build System}: Cargo with custom build scripts
    \item \textbf{Linker Configuration}: Custom linking via linker.ld
  \end{itemize}
\end{frame}

\begin{frame}[fragile]{Key Implementation Concepts}
  \begin{lstlisting}[language=Rust]
// No standard library dependency
#![no_std]
#![allow(incomplete_features)]
#![feature(generic_const_exprs)]

// Core library structure
pub mod macros;
pub mod target;
pub mod file;
pub mod memory;
pub mod traits;
pub mod types;
pub mod result;
  \end{lstlisting}
\end{frame}

\section{Results and Discussion}

\begin{frame}{Memory Management Implementation}
  \begin{itemize}
    \item Implemented a stack parsing system that can:
      \begin{itemize}
        \item Read command-line arguments from stack memory
        \item Safely navigate the stack structure
        \item Extract environment variables
      \end{itemize}
    \item Developed page allocation mechanisms
    \item Created safe wrappers around raw pointers
    \item Implemented trait-based memory operations
  \end{itemize}
\end{frame}

\begin{frame}[fragile]{ELF File Format Handling}
  \begin{itemize}
    \item Implemented ELF header parsing:
      \begin{lstlisting}[language=Rust]
// Example from project
if !arg0.pointer.0.is_null() {
  unsafe {
    let cstr = core::ffi::CStr::from_ptr(
      arg0.pointer.0 as *mut i8);
    let self_path = cstr.to_str().unwrap();
    let identifier =
      userspace::file::format::elf::header::Identifier::from_path(
        self_path);
  }
}
      \end{lstlisting}
    \item Created type-safe representations of ELF structures
    \item Ensured proper endianness handling
  \end{itemize}
\end{frame}

\begin{frame}{Architecture Abstraction}
  \begin{itemize}
    \item Implemented architecture-specific traits:
      \begin{itemize}
        \item Pointer types and operations
        \item Register access
        \item Memory layout definitions
      \end{itemize}
    \item Current focus on x86\_64 architecture
    \item Design allows for easy extension to other architectures
    \item Architecture-specific code isolated in dedicated modules
  \end{itemize}
\end{frame}

\begin{frame}{Cross-Platform Considerations}
  \begin{itemize}
    \item Operating system abstractions:
      \begin{itemize}
        \item System calls
        \item File operations
        \item Process management
      \end{itemize}
    \item Platform detection macros
    \item Feature flags for enabling/disabling functionality
    \item Consistent error handling across platforms
  \end{itemize}
\end{frame}

\begin{frame}{Challenges Encountered}
  \begin{itemize}
    \item Working without the standard library required reimplementation of basic functionality
    \item Handling architecture-specific details while maintaining a clean API
    \item Balancing safety and performance in low-level operations
    \item Creating robust error handling without exceptions
    \item Managing compile-time features and conditional compilation
  \end{itemize}
\end{frame}

\section{Conclusions}

\begin{frame}{Achievements}
  \begin{itemize}
    \item Successfully implemented a foundational userspace library in Rust
    \item Created safe abstractions for low-level system operations
    \item Developed modular architecture for extensibility
    \item Implemented ELF parsing capabilities
    \item Established memory management primitives
    \item Created cross-platform architecture and OS abstractions
  \end{itemize}
\end{frame}

\begin{frame}{Future Work}
  \begin{itemize}
    \item Extend support to additional architectures (ARM, RISC-V)
    \item Implement more comprehensive file system operations
    \item Add networking capabilities
    \item Develop threading and concurrency primitives
    \item Create higher-level abstractions for common operations
    \item Improve documentation and examples
    \item Add comprehensive testing framework
  \end{itemize}
\end{frame}

\begin{frame}{Sustainability Connections (ODS)}
  This project contributes to the following UN Sustainable Development Goals:
  \begin{itemize}
    \item \textbf{Goal 9: Industry, Innovation and Infrastructure}
      \begin{itemize}
        \item Contributing to technological innovation in systems software
        \item Building resilient infrastructure for modern applications
      \end{itemize}
    \item \textbf{Goal 4: Quality Education}
      \begin{itemize}
        \item Open source nature promotes learning and collaboration
        \item Provides educational resources for systems programming
      \end{itemize}
  \end{itemize}
\end{frame}

\begin{frame}{References}
  \begin{thebibliography}{9}
    \bibitem{rustlang} The Rust Programming Language. \url{https://www.rust-lang.org/}

    \bibitem{nostd} The Rust Embedded Book: A \texttt{no\_std} Rust Environment. \url{https://docs.rust-embedded.org/book/intro/no-std.html}

    \bibitem{elf} Tool Interface Standard (TIS) Executable and Linking Format (ELF) Specification.

    \bibitem{userspace} Userspace Project Repository. \url{https://github.com/ze-gois/rust_userspace}

    \bibitem{rustonomicon} The Rustonomicon: The Dark Arts of Advanced and Unsafe Rust Programming. \url{https://doc.rust-lang.org/nomicon/}
  \end{thebibliography}
\end{frame}

\begin{frame}{Acknowledgments}
  \begin{itemize}
    \item Federal University of Rio Grande do Norte (UFRN)
    \item Research advisor and mentors
    \item Rust community for their extensive documentation and support
    \item Open source contributors who inspired this work
  \end{itemize}

  \vspace{1cm}
  \centering
  \large{Thank you!}

  \vspace{0.5cm}
  \small{Contact: ze.gois.00@gmail.com}
\end{frame}

\end{document}
