\section{Padrão TIS (Tool Interface Standard) para ELF}

O formato ELF (Executable and Linkable Format) é definido pelo padrão TIS (Tool Interface Standard), estabelecido pelo Tool Interface Standards Committee. Este padrão define detalhadamente a estrutura e organização dos arquivos binários no formato ELF, permitindo que ferramentas como compiladores, linkers, carregadores e depuradores possam trabalhar de maneira coerente com esses arquivos.

\subsection{Histórico e Desenvolvimento}

O formato ELF foi originalmente desenvolvido pela USL (UNIX System Laboratories) como parte do ABI (Application Binary Interface) do System V Release 4 (SVR4). Em 1993, o comitê TIS (Tool Interface Standard) publicou a especificação ELF oficial, que foi amplamente adotada pelas distribuições Unix e Unix-like, incluindo Linux, BSD, Solaris, HP-UX, entre outros.

Antes do ELF, sistemas Unix utilizavam formatos como a.out e COFF (Common Object File Format), mas o ELF foi projetado para superar as limitações desses formatos anteriores e prover maior flexibilidade e extensibilidade.

\subsection{Objetivos do Padrão TIS}

A especificação TIS para ELF foi desenvolvida com os seguintes objetivos:

\begin{itemize}
    \item Estabelecer um formato de arquivo objeto que pudesse ser usado em diferentes arquiteturas de processadores.
    \item Permitir operações eficientes de linking estático e dinâmico.
    \item Padronizar a interface entre compiladores, assemblers, linkers e sistemas operacionais.
    \item Facilitar a análise e manipulação de binários por ferramentas de desenvolvimento.
    \item Suportar extensão para novas arquiteturas e funcionalidades.
\end{itemize}

\subsection{Alcance da Especificação}

A especificação TIS para ELF define:

\begin{itemize}
    \item Estrutura do cabeçalho ELF (ELF Header)
    \item Formato das tabelas de programa (Program Headers)
    \item Formato das tabelas de seção (Section Headers)
    \item Tipos de seções e seus conteúdos
    \item Tabelas de símbolos e relocações
    \item Convenções para informações de depuração
    \item Extensões específicas de arquitetura
\end{itemize}

\subsection{Implementação em Diferentes Sistemas}

Embora o ELF seja um padrão, diferentes sistemas operacionais e arquiteturas implementam extensões específicas:

\begin{itemize}
    \item \textbf{Linux}: Implementa extensões específicas para suas funcionalidades, como suporte a segurança SELinux e capacidades.
    \item \textbf{FreeBSD/NetBSD}: Adicionam extensões para suas características específicas.
    \item \textbf{Solaris}: Implementa extensões adicionais para suportar funcionalidades específicas da plataforma.
    \item \textbf{ARM/MIPS/PowerPC}: Possuem especificações complementares para lidar com características específicas dessas arquiteturas.
\end{itemize}

\subsection{Documentação TIS}

A especificação TIS para ELF é documentada em vários documentos, incluindo:

\begin{itemize}
    \item Especificação genérica ELF (TIS ELF 1.2)
    \item Especificações específicas de processador (como IA-32, AMD64, ARM, etc.)
    \item ABIs específicos de sistema operacional
\end{itemize}

O documento base, conhecido como "Tool Interface Standard (TIS) Executable and Linkable Format (ELF) Specification", define a estrutura fundamental do formato, enquanto documentos adicionais cobrem extensões e adaptações específicas para diferentes arquiteturas e sistemas.

\subsection{Evolução e Versões}

A especificação ELF evoluiu ao longo do tempo para acomodar novas funcionalidades e arquiteturas:

\begin{itemize}
    \item ELF 1.0: Versão inicial para SVR4
    \item ELF 1.1: Adicionou suporte para executáveis dinâmicos
    \item ELF 1.2: Incluiu extensões e clarificações importantes
    \item Versões posteriores: Extensões para 64 bits, novos processadores e funcionalidades específicas de sistema
\end{itemize}

Em sistemas modernos, a especificação é mantida por várias entidades, incluindo a Linux Foundation, grupos de desenvolvimento BSD, e empresas como Oracle (para Solaris) e ARM.

Esta padronização, embora com variações específicas para diferentes sistemas, permite a portabilidade de binários entre sistemas compatíveis e facilita o desenvolvimento de ferramentas que manipulam arquivos binários em diversos ambientes.